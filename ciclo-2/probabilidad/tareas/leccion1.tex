\documentclass[12pt]{article}


\usepackage[utf8]{inputenc}
\usepackage[spanish]{babel}
\usepackage[margin = 2.54cm]{geometry}
\usepackage{graphicx}
\usepackage{bm}
\usepackage{amsmath}
\usepackage{xcolor}
\usepackage{enumitem}
\usepackage{gensymb}


% ----------------- UTILIDADES PARA DAR UN MEJOR FORMATO DE DOCUMENTO -----------------  


\definecolor{azul}{rgb}{0.0039, 0.3098, 0.6196}


% Formato para el indice general ...........
\makeatletter
    \renewcommand{\@dotsep}{1.5}
    \renewcommand{\l@section}{\@dottedtocline{1}{1.5em}{2.3em}}
    \renewcommand{\l@subsection}{\@dottedtocline{2}{3.8em}{3.2em}}
    \renewcommand{\l@subsubsection}{\@dottedtocline{3}{7.0em}{4.1em}}
\makeatother

% --------- COMANDOS PERSONALIZADOS PARA LA PORTADA DE LAS TAREAS, TRABAJOS Y PROYECTOS ---------

\newcommand{\rutaLogo}[1]{\newcommand{\RutaLogo}{#1}}
\newcommand{\tema}[1]{\newcommand{\Tema}{#1}}
\newcommand{\etiquetaAutores}[1]{\newcommand{\EtiquetaAutores}{#1}}
\newcommand{\alumno}[1]{\newcommand{\Alumno}{#1}}
\newcommand{\materia}[1]{\newcommand{\Materia}{#1}}
\newcommand{\docente}[1]{\newcommand{\Docente}{#1}}
\newcommand{\ciclo}[1]{\newcommand{\Ciclo}{#1}}
\newcommand{\fecha}[1]{\newcommand{\Fecha}{#1}}
\newcommand{\periodo}[1]{\newcommand{\Periodo}{#1}}



\rutaLogo{../../../docs/img/logo-ista.png}
\tema{\\ \vspace{0.8cm} Evaluación interciclo \\ \vspace{1.5cm}}
\etiquetaAutores{Alumno: }
\alumno{Eduardo Mendieta \vspace{0.8cm}}
\materia{Probabilidad y estadística \vspace{0.8cm}}
\docente{Eco. Hermann Seminario \vspace{0.8cm}}
\ciclo{Segundo ciclo \vspace{0.8cm}}
\fecha{19/12/2024 \vspace{0.8cm}}
\periodo{ 2024 - II}


\begin{document}
    \begin{titlepage}

    \centering

    \includegraphics[width=0.11\textwidth]{\RutaLogo} 

    \vspace{0.3cm}
    \textcolor{azul}{\Large \textbf{Instituto Superior Universitario Tecnológico del Azuay \\}}
    \vspace{0.3cm}
    \textcolor{azul}{\Large \textbf{Tecnología Superior en Big Data}}
    
    % 1. ---------------- TEMA -------------------------
    
    {\Large\textbf{\Tema}}
    
    % 2. ---------------- AUTOR(ES) -------------------------
    \textcolor{azul}{\large \textbf{\EtiquetaAutores} \\}
    \vspace{0.3cm}
    {\large \Alumno}

    % 3. ---------------- MATERIA -------------------------
    \textcolor{azul}{\large \textbf{Materia:} \\}
    \vspace{0.3cm}
    {\large \Materia}


    % 3. ---------------- DOCENTE -------------------------
    \textcolor{azul}{\large \textbf{Docente:} \\}
    \vspace{0.3cm}
    {\large \Docente}


    % 3. ---------------- Ciclo -------------------------
    \textcolor{azul}{\large \textbf{Ciclo:} \\}
    \vspace{0.3cm}
    {\large \Ciclo}


    % 3. ---------------- FECHA -------------------------
    \textcolor{azul}{\large \textbf{Fecha:} \\}
    \vspace{0.3cm}
    {\large \Fecha}

    % 3. ---------------- PERIODO -------------------------
    \textcolor{azul}{\large \textbf{Periodo Académico:} \\}
    \vspace{0.3cm}
    {\large \Periodo}
 
\end{titlepage}


    \section*{\centering LECCIÓN} 
    \vspace{0.5cm}\textbf{Encuentre el resultado de los siguientes ejercicios con su respectivo desarrollo:} \vspace{0.5cm}

    \begin{enumerate}[label=\textbf{\arabic*.}]
        % Ejercicio 1: -------------------------------------------------------------
        \item \textbf{Un centro de atención telefónica recibe en promedio 3 llamadas por minuto. ¿Cuál es la probabilidad de que 
        en un minuto determinado reciban exactamente 5 llamadas?}

        \[\mu = 3 \mid x = 5 \mid e = 2.72\]

        \[P(x) = \frac{e^{-\mu}\cdot \mu^x}{x!} = P(5) = \frac{2.72^{-3}\cdot 3^5}{5!}
            = \frac{12.08}{120} = 0.1006 \longrightarrow 10.06\%
        \]

        \texttt{Respuesta:} laprobabilidad de que en un minuto determinado reciban exactamente 5 llamadas
        es del 10.06\%

        % Ejercicio 2: -------------------------------------------------------------
        \item \textbf{La altura de los estudiantes de una escuela sigue una distribución normal con una media de 170 cm y una 
        desviación estándar de 10 cm. ¿Cuál es la probabilidad de que un estudiante seleccionado al azar tenga una altura menor de 160 cm?}

        \[\mu = 170cm \mid x = 160cm \mid \sigma = 10cm\]

        \[z = \frac{x - \mu}{\sigma} = \frac{160 - 170}{10} = \frac{-10}{10} = -1\]

        Buscando $z = -1$ en la tabla, es igual a $1 - 0.8413 = 0.1587$

        \[P(x < 160) = 15.87\%\]

        \texttt{Respuesta:} La probabilidad de que un estudiante tenga una altura menor de 160cm es del 15.87\%

        % Ejercicio 3: -------------------------------------------------------------
        \item \textbf{Los tiempos de entrega de un servicio de mensajería siguen una distribución normal con una media de 30 minutos y 
        una desviación estándar de 5 minutos. ¿Cuál es la probabilidad de que un paquete se entregue en más de 35 minutos?}

        \[\mu = 30min \mid x = 35min \mid \sigma = 5min\]

        \[z = \frac{x - \mu}{\sigma} = \frac{35 - 30}{5} = \frac{5}{5} = 1\]

        Buscando $z = 1$ en la tabla, es igual a $0.8413$

        \[P(x > 35)= 1 - 0.8413 = 0.1587 \longrightarrow 15.87\%\]

        \texttt{Respuesta:} La probabilidad de que un paquete se entregue en más de 35 minutos es del 15.87\%
        
    \end{enumerate}
\end{document}