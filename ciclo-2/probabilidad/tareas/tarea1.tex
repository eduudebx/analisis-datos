\documentclass[12pt]{article}


\usepackage[utf8]{inputenc}
\usepackage[spanish]{babel}
\usepackage[margin = 2.54cm]{geometry}
\usepackage{graphicx}
\usepackage{bm}
\usepackage{amsmath}
\usepackage{xcolor}
\usepackage{enumitem}


% ----------------- UTILIDADES PARA DAR UN MEJOR FORMATO DE DOCUMENTO -----------------  


\definecolor{azul}{rgb}{0.0039, 0.3098, 0.6196}


% Formato para el indice general ...........
\makeatletter
    \renewcommand{\@dotsep}{1.5}
    \renewcommand{\l@section}{\@dottedtocline{1}{1.5em}{2.3em}}
    \renewcommand{\l@subsection}{\@dottedtocline{2}{3.8em}{3.2em}}
    \renewcommand{\l@subsubsection}{\@dottedtocline{3}{7.0em}{4.1em}}
\makeatother

% --------- COMANDOS PERSONALIZADOS PARA LA PORTADA DE LAS TAREAS, TRABAJOS Y PROYECTOS ---------

\newcommand{\rutaLogo}[1]{\newcommand{\RutaLogo}{#1}}
\newcommand{\tema}[1]{\newcommand{\Tema}{#1}}
\newcommand{\etiquetaAutores}[1]{\newcommand{\EtiquetaAutores}{#1}}
\newcommand{\alumno}[1]{\newcommand{\Alumno}{#1}}
\newcommand{\materia}[1]{\newcommand{\Materia}{#1}}
\newcommand{\docente}[1]{\newcommand{\Docente}{#1}}
\newcommand{\ciclo}[1]{\newcommand{\Ciclo}{#1}}
\newcommand{\fecha}[1]{\newcommand{\Fecha}{#1}}
\newcommand{\periodo}[1]{\newcommand{\Periodo}{#1}}



\rutaLogo{../../../docs/img/logo-ista.png}
\tema{\\ \vspace{0.8cm} Taller de ejercicios N\degree1 - Probabilidades \\ \vspace{1.5cm}}
\etiquetaAutores{Alumno: }
\alumno{Eduardo Mendieta \vspace{0.8cm}}
\materia{Probabilidad y estadística \vspace{0.8cm}}
\docente{Eco. Hermann Seminario \vspace{0.8cm}}
\ciclo{Segundo ciclo \vspace{0.8cm}}
\fecha{11/11/2024 \vspace{0.8cm}}
\periodo{ 2024 - II}
\usepackage{gensymb}


\begin{document}
    \begin{titlepage}

    \centering

    \includegraphics[width=0.11\textwidth]{\RutaLogo} 

    \vspace{0.3cm}
    \textcolor{azul}{\Large \textbf{Instituto Superior Universitario Tecnológico del Azuay \\}}
    \vspace{0.3cm}
    \textcolor{azul}{\Large \textbf{Tecnología Superior en Big Data}}
    
    % 1. ---------------- TEMA -------------------------
    
    {\Large\textbf{\Tema}}
    
    % 2. ---------------- AUTOR(ES) -------------------------
    \textcolor{azul}{\large \textbf{\EtiquetaAutores} \\}
    \vspace{0.3cm}
    {\large \Alumno}

    % 3. ---------------- MATERIA -------------------------
    \textcolor{azul}{\large \textbf{Materia:} \\}
    \vspace{0.3cm}
    {\large \Materia}


    % 3. ---------------- DOCENTE -------------------------
    \textcolor{azul}{\large \textbf{Docente:} \\}
    \vspace{0.3cm}
    {\large \Docente}


    % 3. ---------------- Ciclo -------------------------
    \textcolor{azul}{\large \textbf{Ciclo:} \\}
    \vspace{0.3cm}
    {\large \Ciclo}


    % 3. ---------------- FECHA -------------------------
    \textcolor{azul}{\large \textbf{Fecha:} \\}
    \vspace{0.3cm}
    {\large \Fecha}

    % 3. ---------------- PERIODO -------------------------
    \textcolor{azul}{\large \textbf{Periodo Académico:} \\}
    \vspace{0.3cm}
    {\large \Periodo}
 
\end{titlepage}


    \section*{\centering  Taller de ejercicios N\degree1 - Probabilidades} 
    \vspace{0.5cm}\textbf{Resolver los siguientes ejercicios:} \vspace{0.5cm}

    \begin{enumerate}[label=\textbf{\arabic*.}]
        % Ejercicio 1: -------------------------------------------------------------
        \item \textbf{Una urna tiene ocho bolas rojas, cinco amarillas y siete verdes. Si se extrae una bola al azar calcular la 
                        probabilidad de:}

                        \begin{itemize}
                            \item \textbf{Sea roja:} \[\frac{8}{20} = 0.4 = 40\%\]
                            \item \textbf{Sea verde:} \[\frac{7}{20} = 0.35 = 35\%\]
                            \item \textbf{Sea amarilla:} \[\frac{5}{20} = 0.25 = 25\%\]
                            \item \textbf{No sea roja:} \[\frac{12}{20} = 0.6 = 60\%\]
                            \item \textbf{No sea amarilla:} \[\frac{15}{20} = 0.75 = 75\%\]
                        \end{itemize}

        % Ejercicio 2: -------------------------------------------------------------
        \item \textbf{En una clase hay 10 alumnas rubias, 20 morenas, 5 alumnos rubios y 10
                        morenos. Un día asisten 45 alumnos, encontrar la probabilidad de que un
                        alumno:}

                        \begin{itemize}
                            \item \textbf{Sea hombre:}  \[\frac{15}{45} = 0.3333 = 33.33\%\]
                            \item \textbf{Sea mujer morena:} \[\frac{20}{45} = 0.4444 = 44.44 \%\]
                            \item \textbf{Sea hombre o mujer:} \[\frac{45}{45} = 1 = 100\%\]
                        \end{itemize}
        
        % Ejercicio 3: -------------------------------------------------------------
        \item \textbf{En una sala de clases hay 20 mujeres y 12 hombres. Si se escoge uno de ellos al
                        azar. \textquestiondown Cuál es la probabilidad de que la persona escogida sea hombre?}

                         \[\frac{12}{32} = 0.375 = 37.5\%\]
        
        % Ejercicio 4: -------------------------------------------------------------
        \newpage
        \item \textbf{En una comida hay 28 hombres y 32 mujeres. Han comido carne 16 hombres y
                        20 mujeres, comiendo pescado el resto. Si se elige una de las personas al azar.
                        \textquestiondown Cuál es la probabilidad de que?}

                        \begin{itemize}
                            \item \textbf{La persona escogida sea hombre:} \[\frac{28}{60} = 0.4666 = 46.66\%\]
                            \item \textbf{La persona escogida sea mujer y que haya comido carne:} \[\frac{20}{60} = 0.3333 = 33.33\%\]
                        \end{itemize}
        
        % Ejercicio 5: -------------------------------------------------------------
        \item \textbf{La probabilidad de que al sacar una carta al azar de un naipe inglés (52 cartas), una de 
                        ellas sea un AS.}

                        \[\frac{4}{52} = 0.0769 = 7.69\%\]
        
        % Ejercicio 6: -------------------------------------------------------------
        \item \textbf{En un jardín infantil hay 8 morenos y 12 morenas, así como 7 rubios y 5 rubias.
                        Si se elige un integrante al azar, \textquestiondown Cuál es la probabilidad de que sea rubio o
                        rubia ?}

                        \[\frac{12}{32} = 0.375 = 37.5\%\]
        
        % Ejercicio 7: -------------------------------------------------------------
        \item \textbf{En una encuesta, 70 personas prefieren el té, 30 prefieren el café y 50 no tienen
                        preferencia. Si se elige una persona al azar, \textquestiondown cuál es la probabilidad de que
                        prefiera el café? Emite tu criterio sobre las preferencias de la población
                        encuestada.}

                         \[\frac{30}{150} = 0.2 = 20\%\]

                         \texttt{Criterio:} La población claramente tiene más preferencia por el té frente al café, tal vez 
                                            debido a que este se percibe como más saludable, ya que contiene menos cafeína. 
                                            En cambio, al resto de la población no le importa tanto cuál de las dos bebidas elegir.
        
        % Ejercicio 8: -------------------------------------------------------------
        \item \textbf{En una tienda, el 15\% de las prendas están en descuento. Si seleccionas
                        una prenda al azar, \textquestiondown cuál es la probabilidad de que no esté en descuento?}

                        La probabilidad es el $85\%$ restante.
        
        % Ejercicio 9: -------------------------------------------------------------
        \item \textbf{Se estima que el 40\% de los jóvenes utilizan redes sociales para noticias. Si
                        eliges a un joven al azar, \textquestiondown  cuál es la probabilidad de que no utilice redes
                        sociales para este fin?}

                        La probabilidad es el $60\%$ restante.
        
        % Ejercicio 10: ------------------------------------------------------------
        \item \textbf{En una encuesta, el 25\% de las personas dijeron que asistirían a un evento.
                        Si seleccionas a una persona al azar, \textquestiondown  cuál es la probabilidad de que no
                        asista al evento?}

                        La probabilidad es el $75\%$ restante.
        
        % Ejercicio 11: ------------------------------------------------------------
        \item \textbf{En una tienda, hay 10 manzanas, 5 plátanos y 15 naranjas. Si decides
                        comprar una fruta al azar, \textquestiondown cuál es la probabilidad de que compres un
                        plátano?}

                        \[\frac{5}{30} = 0.1667 = 16.67\%\]
        
        % Ejercicio 12: ------------------------------------------------------------
        \item \textbf{En una clase de 30 estudiantes, 18 asisten regularmente y 12 no. Si eliges a
                        un estudiante al azar, \textquestiondown cuál es la probabilidad de que asista regularmente a
                        clase?}

                        \[\frac{18}{30} = 0.6 = 60\%\]
        
        % Ejercicio 13: ------------------------------------------------------------
        \item \textbf{Supón que hay 8 restaurantes en tu barrio, de los cuales 3 son de comida
                        italiana, 2 de comida china y 3 de comida mexicana. Si decides elegir un
                        restaurante al azar para cenar, \textquestiondown cuál es la probabilidad de que elijas un
                        restaurante de comida italiana?}

                        \[\frac{3}{8} = 0.375 = 37.5\%\]
        
        % Ejercicio 14: ------------------------------------------------------------
        \item \textbf{En una bolsa hay 5 bolas rojas, 3 bolas azules y 2 bolas verdes. Si se
                        extraen 2 bolas sin reposición, \textquestiondown cuál es la probabilidad de que ambas sean
                        rojas?}

                        \[\frac{5}{10} \cdot \frac{4}{9} = 0.5 \cdot 0.4444 = 0.2222 = 22.22\%\]
        
        % Ejercicio 15: ------------------------------------------------------------
        \item \textbf{En una baraja de 52 cartas, si se extraen 2 cartas sin reposición, \textquestiondown cuál es la
                        probabilidad de que ambas sean ases?}

                        \[\frac{4}{52} \cdot \frac{3}{51} = 0.0769 \cdot 0.0588 = 0.0452= 4.52\%\]
        
        % Ejercicio 16: ------------------------------------------------------------
        \item \textbf{En una cesta hay 6 manzanas, 4 peras y 5 plátanos. Si se eligen 2 frutas al
                        azar sin reposición, \textquestiondown cuál es la probabilidad de que ambas sean peras?}

                        \[\frac{4}{15} \cdot \frac{3}{14} = 0.2667 \cdot 0.2143 = 0.0572 = 5.72\%\]
        
        % Ejercicio 17: ------------------------------------------------------------
        \item \textbf{En una clase de 20 estudiantes, 8 son mujeres y 12 son hombres. Si se
                        eligen 2 estudiantes sin reposición, \textquestiondown cuál es la probabilidad de que ambos
                        sean hombres?}

                        \[\frac{12}{20} \cdot \frac{11}{19} = 0.6 \cdot 0.5789 = 0.3474 = 34.74\%\]
    \end{enumerate}

\end{document}