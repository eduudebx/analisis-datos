\documentclass[12pt]{article}


\usepackage[utf8]{inputenc}
\usepackage[spanish]{babel}
\usepackage[margin = 2.54cm]{geometry}
\usepackage{graphicx}
\usepackage{bm}
\usepackage{amsmath}
\usepackage{xcolor}
\usepackage{enumitem}
\usepackage{gensymb}


% ----------------- UTILIDADES PARA DAR UN MEJOR FORMATO DE DOCUMENTO -----------------  


\definecolor{azul}{rgb}{0.0039, 0.3098, 0.6196}


% Formato para el indice general ...........
\makeatletter
    \renewcommand{\@dotsep}{1.5}
    \renewcommand{\l@section}{\@dottedtocline{1}{1.5em}{2.3em}}
    \renewcommand{\l@subsection}{\@dottedtocline{2}{3.8em}{3.2em}}
    \renewcommand{\l@subsubsection}{\@dottedtocline{3}{7.0em}{4.1em}}
\makeatother

% --------- COMANDOS PERSONALIZADOS PARA LA PORTADA DE LAS TAREAS, TRABAJOS Y PROYECTOS ---------

\newcommand{\rutaLogo}[1]{\newcommand{\RutaLogo}{#1}}
\newcommand{\tema}[1]{\newcommand{\Tema}{#1}}
\newcommand{\etiquetaAutores}[1]{\newcommand{\EtiquetaAutores}{#1}}
\newcommand{\alumno}[1]{\newcommand{\Alumno}{#1}}
\newcommand{\materia}[1]{\newcommand{\Materia}{#1}}
\newcommand{\docente}[1]{\newcommand{\Docente}{#1}}
\newcommand{\ciclo}[1]{\newcommand{\Ciclo}{#1}}
\newcommand{\fecha}[1]{\newcommand{\Fecha}{#1}}
\newcommand{\periodo}[1]{\newcommand{\Periodo}{#1}}



\rutaLogo{../../../docs/img/logo-ista.png}
\tema{\\ \vspace{0.8cm} Taller de ejercicios N\degree2 - Probabilidades \\ \vspace{1.5cm}}
\etiquetaAutores{Alumno: }
\alumno{Eduardo Mendieta \vspace{0.8cm}}
\materia{Probabilidad y estadística \vspace{0.8cm}}
\docente{Eco. Hermann Seminario \vspace{0.8cm}}
\ciclo{Segundo ciclo \vspace{0.8cm}}
\fecha{20/11/2024 \vspace{0.8cm}}
\periodo{ 2024 - II}


\begin{document}
    \begin{titlepage}

    \centering

    \includegraphics[width=0.11\textwidth]{\RutaLogo} 

    \vspace{0.3cm}
    \textcolor{azul}{\Large \textbf{Instituto Superior Universitario Tecnológico del Azuay \\}}
    \vspace{0.3cm}
    \textcolor{azul}{\Large \textbf{Tecnología Superior en Big Data}}
    
    % 1. ---------------- TEMA -------------------------
    
    {\Large\textbf{\Tema}}
    
    % 2. ---------------- AUTOR(ES) -------------------------
    \textcolor{azul}{\large \textbf{\EtiquetaAutores} \\}
    \vspace{0.3cm}
    {\large \Alumno}

    % 3. ---------------- MATERIA -------------------------
    \textcolor{azul}{\large \textbf{Materia:} \\}
    \vspace{0.3cm}
    {\large \Materia}


    % 3. ---------------- DOCENTE -------------------------
    \textcolor{azul}{\large \textbf{Docente:} \\}
    \vspace{0.3cm}
    {\large \Docente}


    % 3. ---------------- Ciclo -------------------------
    \textcolor{azul}{\large \textbf{Ciclo:} \\}
    \vspace{0.3cm}
    {\large \Ciclo}


    % 3. ---------------- FECHA -------------------------
    \textcolor{azul}{\large \textbf{Fecha:} \\}
    \vspace{0.3cm}
    {\large \Fecha}

    % 3. ---------------- PERIODO -------------------------
    \textcolor{azul}{\large \textbf{Periodo Académico:} \\}
    \vspace{0.3cm}
    {\large \Periodo}
 
\end{titlepage}


    \section*{\centering  Taller de ejercicios N\degree2 - Probabilidades} 
    \vspace{0.5cm}\textbf{Ejercicios sobre reglas de probabilidad:} \vspace{0.5cm}

    \begin{enumerate}[label=\textbf{\arabic*.}]
        % Ejercicio 1: -------------------------------------------------------------
        \item \textbf{En una ciudad, la probabilidad de que llueva en la mañana es 0.4, y la
        probabilidad de que llueva en la tarde es 0.5. Si la probabilidad de que llueva
        tanto en la mañana como en la tarde es 0.2, \textquestiondown cuál es la probabilidad de que
        llueva al menos en una parte del día?}

        % Ejercicio 2: -------------------------------------------------------------
        \item \textbf{La probabilidad de que una persona no conteste su teléfono cuando recibe una
        llamada es 0.15. Si recibe una llamada, \textquestiondown cuál es la probabilidad de que sí
        conteste?}
        
        % Ejercicio 3: -------------------------------------------------------------
        \item \textbf{Un grupo de amigos tiene tres opciones de películas para ver en casa: acción,
        comedia o drama. La probabilidad de que elijan una película de acción es 0.3, la
        probabilidad de que elijan una de comedia es 0.5, y la probabilidad de que les
        guste tanto una película de acción como de comedia es 0.1. \textquestiondown Cuál es la
        probabilidad de que elijan una película de acción o de comedia?}
        
        % Ejercicio 4: -------------------------------------------------------------
        \item \textbf{En un supermercado, la probabilidad de que una persona compre frutas es 0.6,
        y la probabilidad de que compre verduras es 0.5. Si la probabilidad de que
        compre ambas cosas es 0.3, \textquestiondown cuál es la probabilidad de que compre al menos
        frutas o verduras?}
        
        % Ejercicio 5: -------------------------------------------------------------
        \item \textbf{El 70\% de los clientes de una tienda en línea compran productos electrónicos.
        De ellos, el 20\% también compran accesorios. Si un cliente ya compró un
        producto electrónico, \textquestiondown cuál es la probabilidad de que también compre
        accesorios?}
        
        % Ejercicio 6: -------------------------------------------------------------
        \item \textbf{Se lanzan dos dados estándar. \textquestiondown Cuál es la probabilidad de que la suma sea un
        número impar o que salga al menos un número ``6''? (Nota: Algunos resultados
        pueden cumplir ambas condiciones, así que no son eventos mutuamente
        excluyentes).}
        
        % Ejercicio 7: -------------------------------------------------------------
        \item \textbf{En un parque, la probabilidad de que una persona elija practicar fútbol es 0.4, la
        probabilidad de que elija baloncesto es 0.3, y la probabilidad de que elija ambos
        deportes es 0.1. \textquestiondown Cuál es la probabilidad de que una persona elija al menos uno
        de estos deportes?}

        % Ejercicio 8: -------------------------------------------------------------
        \item \textbf{En un parque, la probabilidad de que una persona elija practicar fútbol es 0.4, la
        probabilidad de que elija baloncesto es 0.3, y la probabilidad de que elija ambos
        deportes es 0.1. \textquestiondown Cuál es la probabilidad de que una persona elija al menos uno
        de estos deportes?}
        
        % Ejercicio 9: -------------------------------------------------------------
        \item \textbf{En una semana, la probabilidad de que una persona llegue a tiempo al trabajo
        cada día es 0.9. \textquestiondown Cuál es la probabilidad de que llegue tarde al menos un día?}
        
        % Ejercicio 10: -------------------------------------------------------------
        \item \textbf{En una clase, la probabilidad de que un estudiante no apruebe un examen es
        0.2. \textquestiondown Cuál es la probabilidad de que lo apruebe?}
        
        % Ejercicio 11: -------------------------------------------------------------
        \item \textbf{En una baraja estándar de 52 cartas, la probabilidad de sacar un As es 4/52.
        \textquestiondown Cuál es la probabilidad de no sacar un As en un solo intento?}

        % Ejercicio 12: -------------------------------------------------------------
        \item \textbf{Se lanzan dos monedas. La probabilidad de que una moneda caiga en cara es
        0.5. \textquestiondown Cuál es la probabilidad de que ambas monedas caigan en cara?}
        
        % Ejercicio 13: -------------------------------------------------------------
        \item \textbf{Un amigo tiene dos boletos para una rifa. La probabilidad de ganar con cada
        boleto es 0.05. Si los resultados son independientes, \textquestiondown cuál es la probabilidad de
        que gane con ambos boletos?}
        
        % Ejercicio 14: -------------------------------------------------------------
        \item \textbf{Un estudiante responde al azar 3 preguntas de opción múltiple. La probabilidad
        de acertar cada pregunta es 0.25. Si las respuestas son independientes, \textquestiondown cuál es
        la probabilidad de acertar todas las preguntas?}
        
        % Ejercicio 15: -------------------------------------------------------------
        \item \textbf{Una bombilla tiene una probabilidad de encender correctamente del 0.9. Si se
        encienden dos bombillas independientes, \textquestiondown cuál es la probabilidad de que ambas
        funcionen?}
        
        % Ejercicio 16: -------------------------------------------------------------
        \item \textbf{En una encuesta, el 60\% de las personas prefieren pizza y el 30\% de los que
        prefieren pizza también prefieren hamburguesas. Si se elige a alguien al azar,
        \textquestiondown cuál es la probabilidad de que prefiera hamburguesas dado que prefiere pizza?}
        
        % Ejercicio 17: -------------------------------------------------------------
        \item \textbf{En una universidad, el 30\% de los estudiantes están inscritos en Matemáticas, y
        el 40\% de los inscritos en Matemáticas también están inscritos en Física. \textquestiondown Cuál
        es la probabilidad de que un estudiante esté inscrito en Física dado que ya se
        sabe que está inscrito en Matemáticas?}

    \end{enumerate}
\end{document}