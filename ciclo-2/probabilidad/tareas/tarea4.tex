\documentclass[12pt]{article}


\usepackage[utf8]{inputenc}
\usepackage[spanish]{babel}
\usepackage[margin = 2.54cm]{geometry}
\usepackage{graphicx}
\usepackage{bm}
\usepackage{amsmath}
\usepackage{xcolor}
\usepackage{enumitem}
\usepackage{gensymb}


% ----------------- UTILIDADES PARA DAR UN MEJOR FORMATO DE DOCUMENTO -----------------  


\definecolor{azul}{rgb}{0.0039, 0.3098, 0.6196}


% Formato para el indice general ...........
\makeatletter
    \renewcommand{\@dotsep}{1.5}
    \renewcommand{\l@section}{\@dottedtocline{1}{1.5em}{2.3em}}
    \renewcommand{\l@subsection}{\@dottedtocline{2}{3.8em}{3.2em}}
    \renewcommand{\l@subsubsection}{\@dottedtocline{3}{7.0em}{4.1em}}
\makeatother

% --------- COMANDOS PERSONALIZADOS PARA LA PORTADA DE LAS TAREAS, TRABAJOS Y PROYECTOS ---------

\newcommand{\rutaLogo}[1]{\newcommand{\RutaLogo}{#1}}
\newcommand{\tema}[1]{\newcommand{\Tema}{#1}}
\newcommand{\etiquetaAutores}[1]{\newcommand{\EtiquetaAutores}{#1}}
\newcommand{\alumno}[1]{\newcommand{\Alumno}{#1}}
\newcommand{\materia}[1]{\newcommand{\Materia}{#1}}
\newcommand{\docente}[1]{\newcommand{\Docente}{#1}}
\newcommand{\ciclo}[1]{\newcommand{\Ciclo}{#1}}
\newcommand{\fecha}[1]{\newcommand{\Fecha}{#1}}
\newcommand{\periodo}[1]{\newcommand{\Periodo}{#1}}



\rutaLogo{../../../docs/img/logo-ista.png}
\tema{\\ \vspace{0.8cm} Taller de ejercicios N\degree4 - Distribución binomial \\ \vspace{1.5cm}}
\etiquetaAutores{Alumno: }
\alumno{Eduardo Mendieta \vspace{0.8cm}}
\materia{Probabilidad y estadística \vspace{0.8cm}}
\docente{Eco. Hermann Seminario \vspace{0.8cm}}
\ciclo{Segundo ciclo \vspace{0.8cm}}
\fecha{06/12/2024 \vspace{0.8cm}}
\periodo{ 2024 - II}


\begin{document}
    \begin{titlepage}

    \centering

    \includegraphics[width=0.11\textwidth]{\RutaLogo} 

    \vspace{0.3cm}
    \textcolor{azul}{\Large \textbf{Instituto Superior Universitario Tecnológico del Azuay \\}}
    \vspace{0.3cm}
    \textcolor{azul}{\Large \textbf{Tecnología Superior en Big Data}}
    
    % 1. ---------------- TEMA -------------------------
    
    {\Large\textbf{\Tema}}
    
    % 2. ---------------- AUTOR(ES) -------------------------
    \textcolor{azul}{\large \textbf{\EtiquetaAutores} \\}
    \vspace{0.3cm}
    {\large \Alumno}

    % 3. ---------------- MATERIA -------------------------
    \textcolor{azul}{\large \textbf{Materia:} \\}
    \vspace{0.3cm}
    {\large \Materia}


    % 3. ---------------- DOCENTE -------------------------
    \textcolor{azul}{\large \textbf{Docente:} \\}
    \vspace{0.3cm}
    {\large \Docente}


    % 3. ---------------- Ciclo -------------------------
    \textcolor{azul}{\large \textbf{Ciclo:} \\}
    \vspace{0.3cm}
    {\large \Ciclo}


    % 3. ---------------- FECHA -------------------------
    \textcolor{azul}{\large \textbf{Fecha:} \\}
    \vspace{0.3cm}
    {\large \Fecha}

    % 3. ---------------- PERIODO -------------------------
    \textcolor{azul}{\large \textbf{Periodo Académico:} \\}
    \vspace{0.3cm}
    {\large \Periodo}
 
\end{titlepage}


    \section*{\centering  Taller de ejercicios N\degree4 - Distribución binomial} 
    \vspace{0.5cm}\textbf{Resolver los siguientes ejercicios:} \vspace{0.5cm}

    \begin{enumerate}[label=\textbf{\arabic*.}]
        % Ejercicio 1: -------------------------------------------------------------
        \item \textbf{La última novela de un autor ha tenido un gran éxito, hasta el punto de que el 80\% de
        los lectores ya la han leído. \\
        Un grupo de 4 amigos son aficionados a la lectura:}

        \begin{enumerate}[label=\textbf{\alph*.}]
            \item \textbf{¿Cuál es la probabilidad de que en el grupo hayan leído la novela 2 personas?}
            \[p = 0.8 \mid q = 0.2 \mid n = 4 \mid x = 2\]
            \[C^{n}_{x} = \frac{4!}{2!(4-2)!} = \frac{4\cdot 3\cdot 2!}{2!\cdot 2!} = 6\]
            \[x \mapsto B(n, p) = 6 \cdot 0.8^{2} \cdot 0.2^{4-2} = 6 \cdot 0.64 \cdot 0.04 = 0.1535 \rightarrow 15.35\%\]

            \texttt{Respuesta:} La probabilidad de que en el grupo hayan leído la novela 2 personas es del 15.35\%.

            \item \textbf{¿Y cómo máximo 2?}
            \[x = 0\]
            \[C^{n}_{x} = \frac{4!}{0!(4-0)!} = \frac{4!}{4!} = 1\]
            \[x \mapsto B(n, p) = 1 \cdot 0.8^{0} \cdot 0.2^{4-0} = 1 \cdot 1 \cdot 0.0016 = 0.0016 \rightarrow 0.16\%\]

            \[x = 1\]
            \[C^{n}_{x} = \frac{4!}{1!(4-1)!} = \frac{4\cdot 3!}{3!} = 4\]
            \[x \mapsto B(n, p) = 4 \cdot 0.8^{1} \cdot 0.2^{4-1} = 4 \cdot 0.8 \cdot 0.008 = 0.0256 \rightarrow 2.56\%\]

            \[x = 0, 1, 2\]
            \[x \mapsto B(n, p) = 0.0016 + 0.0256 +0.1535 = 0.1807 = 18.07\%\]

            \texttt{Respuesta:} La probabilidad de que en el grupo hayan leído la novela cómo máximo 2 personas es del 18.07\%.
            
        \end{enumerate}


        % Ejercicio 2: -------------------------------------------------------------
        \newpage
        \item \textbf{Un agente de seguros vende pólizas a cinco personas de la misma edad y que disfrutan
        de buena salud. Según las tablas actuales, la probabilidad de que una persona en estas
        condiciones viva 30 años o más es 2/3.\\
        Hállese la probabilidad de que, transcurridos 30 años, vivan:}

        \begin{enumerate}[label=\textbf{\alph*.}]
            \item \textbf{Las cinco personas:}
            \[p = 0.6667 \mid q = 0.3333 \mid n = 5 \mid x = 5\]
            \[C^{n}_{x} = \frac{5!}{5!(5-5)!} = \frac{5!}{5!} = 1\]
            \[x \mapsto B(n, p) = 1 \cdot 0.6667^{5} \cdot 0.3333^{5-5} = 1 \cdot 0.1317 \cdot 1 = 0.1317 \rightarrow 13.17\%\]
            
            \texttt{Respuesta:} La probabilidad de que, transcurridos 30 años, vivan las cinco personas es del 13.17\%.

            \item \textbf{Al menos tres personas:}
            \[x = 2\]
            \[C^{n}_{x} = \frac{5!}{2!(5-2)!} = \frac{5!}{2!\cdot 3!} = \frac{5 \cdot 4\cdot 3!}{2\cdot 3!} = 10\]
            \[x \mapsto B(n, p) = 10 \cdot 0.6667^{2} \cdot 0.3333^{5-2} = 10 \cdot 0.4445 \cdot 0.037 = 0.1645 \rightarrow 16.45\%\]

            \[x = 1\]
            \[C^{n}_{x} = \frac{5!}{1!(5-1)!} = \frac{5!}{4!} = \frac{5 \cdot 4!}{4!} = 5\]
            \[x \mapsto B(n, p) = 5 \cdot 0.6667^{1} \cdot 0.3333^{5-1} = 5 \cdot 0.6667 \cdot 0.0411 =  \rightarrow 4.11\%\]

            \[x = 0\]
            \[C^{n}_{x} = \frac{5!}{0!(5-0)!} = \frac{5!}{5!} = 1\]
            \[x \mapsto B(n, p) = 1 \cdot 0.6667^{0} \cdot 0.3333^{5-0} = 1 \cdot 1 \cdot 0.0041 =  \rightarrow 0.41\%\]

            \[x = 3, 4, 5\]
            
            \[x \mapsto B(n, p) = 1 - 0.1645 - 0.0411 - 0.0041 = 0.7903 \rightarrow 79.03\%\]
            
            \texttt{Respuesta:} La probabilidad de que, transcurridos 30 años, vivan al menos tres personas es del 79.03\%.

            \item \textbf{Exactamente dos personas:}
            \[x = 2\]
            \[x \mapsto B(n, p) = 0.1645 \rightarrow 16.45\%\]

            \texttt{Respuesta:} La probabilidad de que, transcurridos 30 años, vivan exactamente dos personas es del 16.45\%.

            
        \end{enumerate}


        % Ejercicio 3: -------------------------------------------------------------
        \item \textbf{El número de pinchazos en los neumáticos de cierto vehículo industrial tiene una
        media de 0,3 por cada 50000 kilómetros. Si el vehículo recorre 100000 km, calcula la
        probabilidad de que tenga menos de 3 pinchazos.}

        


        % Ejercicio 4: -------------------------------------------------------------
        \item \textbf{En una línea de producción, se sabe que el 2\% de los productos son defectuosos. Si se
        revisan 50 productos, ¿cuál es la probabilidad de que exactamente 3 sean
        defectuosos?}
        
        \[p = 0.02 \mid q = 0.98 \mid n = 50 \mid x = 3\]
            \[C^{n}_{x} = \frac{50!}{3!(50-3)!} = \frac{50\cdot 49\cdot 48\cdot 47!}{6\cdot 47!} = 50\cdot 49\cdot 8 = 19600\]
            \[x \mapsto B(n, p) = 19600 \cdot 0.02^{3} \cdot 0.98^{50-3} = 19600 \cdot 0.000008 \cdot 0.3869 = 0.0607 \rightarrow 6.07\%\]
            
            \texttt{Respuesta:} La probabilidad de que exactamente 3 productos sean defectuosos es del 6.07\%.


        % Ejercicio 5: -------------------------------------------------------------
        \item \textbf{En una encuesta, el 80\% de los clientes están satisfechos con un servicio. Si se encuesta
        a 15 clientes, ¿cuál es la probabilidad de que al menos 12 estén satisfechos?}

        \[p = 0.8 \mid q = 0.2 \mid n = 15 \mid x = 12\]
            \[C^{n}_{x} = \frac{15!}{12!(15-12)!} = \frac{15\cdot 14\cdot 13\cdot 12!}{12!\cdot 3!} = \frac{2700}{6} = 455\]
            \[x \mapsto B(n, p) = 455 \cdot 0.8^{12} \cdot 0.2^{15-12} = 455 \cdot 0.0687 \cdot 0.008 = 0.2501 \rightarrow 25.01\%\]

            \[x = 13\]
            \[C^{n}_{x} = \frac{15!}{13!(15-13)!} = \frac{15\cdot 14\cdot 13!}{13!\cdot 2!} = \frac{210}{2} = 105\]
            \[x \mapsto B(n, p) = 105 \cdot 0.8^{13} \cdot 0.2^{15-13} = 105 \cdot 0.0549 \cdot 0.04 = 0.2309 \rightarrow 23.09\%\]

            \[x = 14\]
            \[C^{n}_{x} = \frac{15!}{14!(15-14)!} = \frac{15\cdot 14!}{14!\cdot 1!} = 15\]
            \[x \mapsto B(n, p) = 15 \cdot 0.8^{14} \cdot 0.2^{15-14} = 15 \cdot 0.04398 \cdot 0.2 = 0.1319 \rightarrow 13.19\%\]

            \[x = 15\]
            \[C^{n}_{x} = \frac{15!}{15!(15-15)!} = \frac{15!}{15!} = 1\]
            \[x \mapsto B(n, p) = 1 \cdot 0.8^{15} \cdot 0.2^{15-15} = 1 \cdot 0.0352 \cdot 1 = 0.0352 \rightarrow 3.52\%\]

            \[x = 12, 13, 14, 15\]
            \[x \mapsto B(n, p) = 0.2501 + 0.2309 + 0.1319 + 0.0352 = 0.6481 \rightarrow 64.81\%\]
            
            \texttt{Respuesta:} La probabilidad de que al menos 12 clientes estén satisfechos es del 64.81\%.


        % Ejercicio 6: -------------------------------------------------------------
        \item \textbf{Un estudiante tiene un 70\% de probabilidad de aprobar un examen. Si el estudiante
        presenta 5 exámenes, ¿cuál es la probabilidad de que apruebe exactamente 4?}

            \[p = 0.7 \mid q = 0.3 \mid n = 5 \mid x = 4\]
            \[C^{n}_{x} = \frac{5!}{4!(5-4)!} = \frac{5\cdot 4!}{4!} = 5\]
            \[x \mapsto B(n, p) = 5 \cdot 0.7^{4} \cdot 0.3^{5-4} = 5 \cdot 0.2401 \cdot 0.3 = 0.3602 \rightarrow 36.02\%\]

            \texttt{Respuesta:} La probabilidad de que apruebe exactamente 4 exámenes es del 36.02\%.
        
        % Ejercicio 7: -------------------------------------------------------------
        \item \textbf{Un arquero tiene una tasa de acierto del 60\%. Si dispara 8 flechas, ¿cuál es la
        probabilidad de que acierte exactamente 5?}

        \[p = 0.6 \mid q = 0.4 \mid n = 8 \mid x = 5\]
            \[C^{n}_{x} = \frac{8!}{5!(8-5)!} = \frac{8\cdot 7\cdot 6\cdot 5!}{5!\cdot 3!} = 56\]
            \[x \mapsto B(n, p) = 56 \cdot 0.6^{5} \cdot 0.4^{8-5} = 56 \cdot 0.0778 \cdot 0.064 = 0.2788 \rightarrow 27.88\%\]

            \texttt{Respuesta:} La probabilidad de que acierte exactamente 5 flechas es del 27.88\%.

    \end{enumerate}
\end{document}