\documentclass[12pt]{article}


\usepackage[utf8]{inputenc}
\usepackage[spanish]{babel}
\usepackage[margin = 2.54cm]{geometry}
\usepackage{graphicx}
\usepackage{bm}
\usepackage{amsmath}
\usepackage{xcolor}
\usepackage{enumitem}
\usepackage{gensymb}


% ----------------- UTILIDADES PARA DAR UN MEJOR FORMATO DE DOCUMENTO -----------------  


\definecolor{azul}{rgb}{0.0039, 0.3098, 0.6196}


% Formato para el indice general ...........
\makeatletter
    \renewcommand{\@dotsep}{1.5}
    \renewcommand{\l@section}{\@dottedtocline{1}{1.5em}{2.3em}}
    \renewcommand{\l@subsection}{\@dottedtocline{2}{3.8em}{3.2em}}
    \renewcommand{\l@subsubsection}{\@dottedtocline{3}{7.0em}{4.1em}}
\makeatother

% --------- COMANDOS PERSONALIZADOS PARA LA PORTADA DE LAS TAREAS, TRABAJOS Y PROYECTOS ---------

\newcommand{\rutaLogo}[1]{\newcommand{\RutaLogo}{#1}}
\newcommand{\tema}[1]{\newcommand{\Tema}{#1}}
\newcommand{\etiquetaAutores}[1]{\newcommand{\EtiquetaAutores}{#1}}
\newcommand{\alumno}[1]{\newcommand{\Alumno}{#1}}
\newcommand{\materia}[1]{\newcommand{\Materia}{#1}}
\newcommand{\docente}[1]{\newcommand{\Docente}{#1}}
\newcommand{\ciclo}[1]{\newcommand{\Ciclo}{#1}}
\newcommand{\fecha}[1]{\newcommand{\Fecha}{#1}}
\newcommand{\periodo}[1]{\newcommand{\Periodo}{#1}}



\rutaLogo{../../../docs/img/logo-ista.png}
\tema{\\ \vspace{0.8cm} Taller de ejercicios N\degree3 - Combinatoria \\ \vspace{1.5cm}}
\etiquetaAutores{Alumno: }
\alumno{Eduardo Mendieta \vspace{0.8cm}}
\materia{Probabilidad y estadística \vspace{0.8cm}}
\docente{Eco. Hermann Seminario \vspace{0.8cm}}
\ciclo{Segundo ciclo \vspace{0.8cm}}
\fecha{24/11/2024 \vspace{0.8cm}}
\periodo{ 2024 - II}


\begin{document}
    \begin{titlepage}

    \centering

    \includegraphics[width=0.11\textwidth]{\RutaLogo} 

    \vspace{0.3cm}
    \textcolor{azul}{\Large \textbf{Instituto Superior Universitario Tecnológico del Azuay \\}}
    \vspace{0.3cm}
    \textcolor{azul}{\Large \textbf{Tecnología Superior en Big Data}}
    
    % 1. ---------------- TEMA -------------------------
    
    {\Large\textbf{\Tema}}
    
    % 2. ---------------- AUTOR(ES) -------------------------
    \textcolor{azul}{\large \textbf{\EtiquetaAutores} \\}
    \vspace{0.3cm}
    {\large \Alumno}

    % 3. ---------------- MATERIA -------------------------
    \textcolor{azul}{\large \textbf{Materia:} \\}
    \vspace{0.3cm}
    {\large \Materia}


    % 3. ---------------- DOCENTE -------------------------
    \textcolor{azul}{\large \textbf{Docente:} \\}
    \vspace{0.3cm}
    {\large \Docente}


    % 3. ---------------- Ciclo -------------------------
    \textcolor{azul}{\large \textbf{Ciclo:} \\}
    \vspace{0.3cm}
    {\large \Ciclo}


    % 3. ---------------- FECHA -------------------------
    \textcolor{azul}{\large \textbf{Fecha:} \\}
    \vspace{0.3cm}
    {\large \Fecha}

    % 3. ---------------- PERIODO -------------------------
    \textcolor{azul}{\large \textbf{Periodo Académico:} \\}
    \vspace{0.3cm}
    {\large \Periodo}
 
\end{titlepage}


    \section*{\centering  Taller de ejercicios N\degree3 - Combinatoria} 
    \vspace{0.5cm}\textbf{Problemas de combinaciones y permutaciones:} \vspace{0.5cm}

    \begin{enumerate}[label=\textbf{\arabic*.}]
        % Ejercicio 1: -------------------------------------------------------------
        \item \textbf{Tienes 8 tipos diferentes de frutas y quieres elegir 3 tipos para hacer una
        ensalada.\\
        Pregunta: ¿De cuántas maneras puedes seleccionar 3 frutas diferentes?}

        El orden de selección de las frutas no importa: $n = 8$, $r = 3$\\

        \[C(8, 3) = \frac{8!}{3!(8 - 3)!} = \frac{8\cdot 7\cdot 6\cdot 5!}{3!\cdot 5!} =  \frac{8\cdot 7\cdot 6}{6} = 56\]

        \texttt{Respuesta:} Existen 56 maneras diferentes de seleccionar 3 frutas de las 8.



        % Ejercicio 2: -------------------------------------------------------------
        \item \textbf{En tu equipo de trabajo hay 10 personas, y necesitas formar un grupo de 4
        personas para un proyecto.\\
        Pregunta: ¿De cuántas maneras puedes elegir las 4 personas?}

        El orden de selección de las personas no importa: $n = 10$, $r = 4$\\

        \[C(10, 4) = \frac{10!}{4!(10 - 4)!} = \frac{10\cdot 9\cdot 8\cdot 7\cdot 6!}{4!\cdot 6!} =  \frac{10\cdot 9\cdot 8\cdot 7}{4\cdot 3\cdot 2\cdot 1} 
        = \frac{5040}{24} = 210\]

        \texttt{Respuesta:} Existen 210 maneras diferentes de elegir 4 personas de las 10.
        


        % Ejercicio 3: -------------------------------------------------------------
        \item \textbf{Tienes una lista de 12 películas y decides ver 5 de ellas durante un fin de
        semana, sin importar el orden.\\
        Pregunta: ¿De cuántas maneras puedes seleccionar las películas?}

        El orden de selección de las películas no importa: $n = 12$, $r = 5$\\

        \[C(12, 5) = \frac{12!}{5!(12 - 5)!} = \frac{12\cdot 11\cdot 10\cdot 9\cdot 8\cdot 7!}{5!\cdot 7!} =  
        \frac{12\cdot 11\cdot 10\cdot 9\cdot 8}{5\cdot 4\cdot 3\cdot 2\cdot 1}  = \frac{95040}{120} = 792\]

        \texttt{Respuesta:} Existen 792 maneras diferentes de elegir 5 películas de las 12.
        


        % Ejercicio 4: -------------------------------------------------------------
        \item \textbf{Hay 6 ponentes programados para una conferencia. El orden de las
        presentaciones importa.\\
        Pregunta: ¿De cuántas maneras pueden ordenarse los 6 ponentes?}

        El orden en el que los ponentes se presentan sí importa.: $n = 6$, $r = 6$\\

        \[P(6, 6) = \frac{6!}{(6 - 6)!} = 6! = 6\cdot 5\cdot 4\cdot 3\cdot 2\cdot 1 = 720\]

        \texttt{Respuesta:} Existen 720 maneras diferentes de ordenar a los 6 ponentes.
        


        % Ejercicio 5: -------------------------------------------------------------
        \item \textbf{Tienes 10 libros diferentes y quieres organizarlos en un estante en un orden
        específico.\\
        Pregunta: ¿De cuántas maneras puedes organizar los 10 libros?}

        El orden en el que se organizan los libros sí importa.: $n = 10$, $r = 10$\\

        \[P(10, 10) = \frac{10!}{(10 - 10)!} = 10! = 10\cdot 9\cdot 8\cdot 7\cdot 6\cdot 5\cdot 4\cdot 3\cdot 2\cdot 1 = 3628800\]

        \texttt{Respuesta:} Existen 3628800 maneras diferentes de ordenar los libros.
        
        


        % Ejercicio 6: -------------------------------------------------------------
        \item \textbf{Una contraseña requiere 3 letras diferentes tomadas del alfabeto (26 letras), y
        el orden importa.\\
        Pregunta: ¿De cuántas maneras puedes crear la contraseña?}

        El orden en el que se toman las letras sí importa.: $n = 26$, $r = 3$\\

        \[P(26, 3) = \frac{26!}{(26 - 3)!} = \frac{26\cdot 25 \cdot 24\cdot 23!}{23!} = 
        26\cdot 25\cdot 24 = 15600\]

        \texttt{Respuesta:} Existen 15600 maneras diferentes de ordenar las letras.

        
        
        % Ejercicio 7: -------------------------------------------------------------
        \item \textbf{En un concurso hay 5 finalistas, y se otorgan premios para el 1.º, 2.º y 3.º lugar,
        considerando el orden.\\
        Pregunta: ¿De cuántas maneras se pueden asignar los premios?}

        El orden de los finalistas sí importa.: $n = 5$, $r = 3$\\

        \[P(5, 3) = \frac{5!}{(5 - 3)!} = \frac{5\cdot 4 \cdot 3\cdot 2!}{2!} = 5\cdot 4 \cdot 3 = 60\]

        \texttt{Respuesta:} Existen 60 maneras diferentes de ordenar a los finalistas.

        

        % Ejercicio 8: -------------------------------------------------------------
        \item \textbf{Una contraseña requiere 3 letras diferentes tomadas del alfabeto (26 letras), y el
        orden importa.\\
        Pregunta: ¿De cuántas maneras puedes crear la contraseña?}

        El orden en el que se toman las letras sí importa.: $n = 26$, $r = 3$\\

        \[P(26, 3) = \frac{26!}{(26 - 3)!} = \frac{26\cdot 25 \cdot 24\cdot 23!}{23!} = 
        26\cdot 25\cdot 24 = 15600\]

        \texttt{Respuesta:} Existen 15600 maneras diferentes de ordenar las letras.


        
        % Ejercicio 9: -------------------------------------------------------------
        \item \textbf{Un estudiante tiene 4 camisas diferentes, 3 pares de pantalones y 2 pares de
        zapatos. Quiere saber cuántos conjuntos diferentes puede formar.\\
        Pregunta: ¿De cuántas maneras diferentes puede combinar una camisa, un
        pantalón y un par de zapatos?}

        Aplicamos la regla del producto de la combinatoria: $n_1 = 4$ camisas, $n_2 = 3$ pares de pantalones y $n_3 = 2$ pares de
        zapatos.

        \[P(n_1, n_2, n_3) = 4 \cdot 3 \cdot 2 = 24\]

        \texttt{Respuesta:} Se pueden combinar de 24 maneras diferentes.


        
        % Ejercicio 10: -------------------------------------------------------------
        \item \textbf{Para una contraseña, se necesita:}
            \begin{itemize}
                \item \textbf{1 letra (de 26 posibles letras del alfabeto).}
                \item \textbf{1 número (del 0 al 9).}
                \item \textbf{1 símbolo especial (de un conjunto de 5 símbolos).}
            \end{itemize}
            \textbf{Pregunta: ¿Cuántas contraseñas diferentes pueden generarse combinando una
            letra, un número y un símbolo especial?}

            Aplicamos la regla del producto de la combinatoria: $n_1 = 26$ posibles letras, $n_2 = 10$ posibles números y $n_3 = 5$ posibles números símbolos.

            \[P(n_1, n_2, n_3) = 26 \cdot 10 \cdot 5 = 1300\]

            \texttt{Respuesta:} Se pueden generar 1300 contraseñas diferentes.
        


        % Ejercicio 11: -------------------------------------------------------------
        \item \textbf{En un restaurante, un cliente puede elegir:}
            \begin{itemize}
                \item \textbf{3 tipos de entrada (sopa, ensalada, o aperitivo).}
                \item \textbf{5 tipos de plato principal.}
                \item \textbf{4 tipos de postre.}
            \end{itemize}
            \textbf{Pregunta: ¿Cuántas combinaciones diferentes de menú (entrada, plato principal
            y postre) puede elegir un cliente?}

            Aplicamos la regla del producto de la combinatoria: $n_1 = 3$ tipos de entrada, $n_2 = 5$ tipos de plato principal y 
            $n_3 = 4$ tipos de postre.

            \[P(n_1, n_2, n_3) = 3 \cdot 5 \cdot 4 = 60\]

            \texttt{Respuesta:} Se pueden generar 60 combinaciones diferentes.

    \end{enumerate}
\end{document}