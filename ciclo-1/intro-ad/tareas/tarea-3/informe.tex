\documentclass[12pt]{article}


% -------------------- PAQUETES --------------------
\usepackage[utf8]{inputenc}
\usepackage[spanish]{babel}
\usepackage[margin=2.54cm]{geometry}
\usepackage{graphicx}
\usepackage{xcolor}
\usepackage{enumitem}
\usepackage{parskip}
\usepackage{hyperref}
\usepackage{ulem} 
\usepackage{subcaption}


% -------------------- CARGA DE ARCHIVOS EXTERNOS --------------------
% ----------------- UTILIDADES PARA DAR UN MEJOR FORMATO DE DOCUMENTO -----------------  


\definecolor{azul}{rgb}{0.0039, 0.3098, 0.6196}


% Formato para el indice general ...........
\makeatletter
    \renewcommand{\@dotsep}{1.5}
    \renewcommand{\l@section}{\@dottedtocline{1}{1.5em}{2.3em}}
    \renewcommand{\l@subsection}{\@dottedtocline{2}{3.8em}{3.2em}}
    \renewcommand{\l@subsubsection}{\@dottedtocline{3}{7.0em}{4.1em}}
\makeatother

% --------- COMANDOS PERSONALIZADOS PARA LA PORTADA DE LAS TAREAS, TRABAJOS Y PROYECTOS ---------

\newcommand{\rutaLogo}[1]{\newcommand{\RutaLogo}{#1}}
\newcommand{\tema}[1]{\newcommand{\Tema}{#1}}
\newcommand{\etiquetaAutores}[1]{\newcommand{\EtiquetaAutores}{#1}}
\newcommand{\alumno}[1]{\newcommand{\Alumno}{#1}}
\newcommand{\materia}[1]{\newcommand{\Materia}{#1}}
\newcommand{\docente}[1]{\newcommand{\Docente}{#1}}
\newcommand{\ciclo}[1]{\newcommand{\Ciclo}{#1}}
\newcommand{\fecha}[1]{\newcommand{\Fecha}{#1}}
\newcommand{\periodo}[1]{\newcommand{\Periodo}{#1}}



% -------------------- DEFINICIÓN DE LA PORTADA --------------------
\rutaLogo{../../../../docs/img/logo-ista.png}
\tema{\\ \vspace{0.5cm} Aplicación del proceso de Análisis de Datos a un dataset de ventas de llantas \\ \vspace{1.2cm}}
\etiquetaAutores{Integrantes:}
\alumno{Nube Gutierrez\\Eduardo Mendieta\vspace{0.7cm}}
\materia{Introducción a Big Data \vspace{0.7cm}}
\docente{MSc. Ing. Carmen Tacuri Vintimilla \vspace{0.7cm}}
\ciclo{Primer Ciclo \vspace{0.7cm}}
\fecha{17 de agosto de 2024 \vspace{0.7cm}}
\periodo{Abril 2024 - Agosto 2024}

\begin{document}

    \begin{titlepage}

    \centering

    \includegraphics[width=0.11\textwidth]{\RutaLogo} 

    \vspace{0.3cm}
    \textcolor{azul}{\Large \textbf{Instituto Superior Universitario Tecnológico del Azuay \\}}
    \vspace{0.3cm}
    \textcolor{azul}{\Large \textbf{Tecnología Superior en Big Data}}
    
    % 1. ---------------- TEMA -------------------------
    
    {\Large\textbf{\Tema}}
    
    % 2. ---------------- AUTOR(ES) -------------------------
    \textcolor{azul}{\large \textbf{\EtiquetaAutores} \\}
    \vspace{0.3cm}
    {\large \Alumno}

    % 3. ---------------- MATERIA -------------------------
    \textcolor{azul}{\large \textbf{Materia:} \\}
    \vspace{0.3cm}
    {\large \Materia}


    % 3. ---------------- DOCENTE -------------------------
    \textcolor{azul}{\large \textbf{Docente:} \\}
    \vspace{0.3cm}
    {\large \Docente}


    % 3. ---------------- Ciclo -------------------------
    \textcolor{azul}{\large \textbf{Ciclo:} \\}
    \vspace{0.3cm}
    {\large \Ciclo}


    % 3. ---------------- FECHA -------------------------
    \textcolor{azul}{\large \textbf{Fecha:} \\}
    \vspace{0.3cm}
    {\large \Fecha}

    % 3. ---------------- PERIODO -------------------------
    \textcolor{azul}{\large \textbf{Periodo Académico:} \\}
    \vspace{0.3cm}
    {\large \Periodo}
 
\end{titlepage}


    \tableofcontents
    \newpage

    \section*{\centering Aplicación del proceso de Análisis de Datos a un dataset de ventas de llantas}

    % 1.Introducción: ...................................................
    \section{Introducción}
                

    % 2.Objetivos: ...................................................
    \newpage
    \section{Objetivos}
        \subsection{Objetivo general}
                        

        \subsection{Objetivos específicos}
            \begin{itemize}
                \item 
            \end{itemize}


    % 3.Paso a Paso: ...................................................
    \newpage
    \section{Paso a paso}
        \subsection{Preparación}
        \subsection{Organización}
        \subsection{Limpieza}
        \subsection{Transformación}
        \subsection{Visualización}


    % 4.Conclusiones: ...................................................
    \newpage
    \section{Conclusiones}
        \begin{enumerate}
            \item 
        \end{enumerate}


    % 5.Bibliografía: ...................................................
    \section{Bibliografía}
        \begin{itemize}
            \item 
        \end{itemize}


\end{document}