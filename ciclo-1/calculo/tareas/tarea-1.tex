\documentclass[12pt]{article}


% ----------------- PAQUETES -----------------
\usepackage[utf8]{inputenc}
\usepackage[spanish]{babel}
\usepackage[margin = 2.54cm]{geometry}
\usepackage{graphicx}
\usepackage{xcolor}
\usepackage{enumitem}
\usepackage{parskip}

% ----------------- ARCHIVOS EXTERNOS -----------------
% ----------------- UTILIDADES PARA DAR UN MEJOR FORMATO DE DOCUMENTO -----------------  


\definecolor{azul}{rgb}{0.0039, 0.3098, 0.6196}


% Formato para el indice general ...........
\makeatletter
    \renewcommand{\@dotsep}{1.5}
    \renewcommand{\l@section}{\@dottedtocline{1}{1.5em}{2.3em}}
    \renewcommand{\l@subsection}{\@dottedtocline{2}{3.8em}{3.2em}}
    \renewcommand{\l@subsubsection}{\@dottedtocline{3}{7.0em}{4.1em}}
\makeatother

% --------- COMANDOS PERSONALIZADOS PARA LA PORTADA DE LAS TAREAS, TRABAJOS Y PROYECTOS ---------

\newcommand{\rutaLogo}[1]{\newcommand{\RutaLogo}{#1}}
\newcommand{\tema}[1]{\newcommand{\Tema}{#1}}
\newcommand{\etiquetaAutores}[1]{\newcommand{\EtiquetaAutores}{#1}}
\newcommand{\alumno}[1]{\newcommand{\Alumno}{#1}}
\newcommand{\materia}[1]{\newcommand{\Materia}{#1}}
\newcommand{\docente}[1]{\newcommand{\Docente}{#1}}
\newcommand{\ciclo}[1]{\newcommand{\Ciclo}{#1}}
\newcommand{\fecha}[1]{\newcommand{\Fecha}{#1}}
\newcommand{\periodo}[1]{\newcommand{\Periodo}{#1}}



% ----------------- PORTADA -----------------
\rutaLogo{../../../docs/img/logo-ista.png}
\tema{\\ \vspace{0.8cm} Taller de ejercicios - Límites \\ \vspace{1.5cm}}
\etiquetaAutores{Alumno: }
\alumno{Eduardo Mendieta \vspace{0.8cm}}
\materia{Matemática \vspace{0.8cm}}
\docente{Lcda. Vilma Duchi, Mgtr. \vspace{0.8cm}}
\ciclo{Primer ciclo \vspace{0.8cm}}
\fecha{05/08/2024 \vspace{0.8cm}}
\periodo{Abril 2024 - Agosto 2024}


\begin{document}
    \begin{titlepage}

    \centering

    \includegraphics[width=0.11\textwidth]{\RutaLogo} 

    \vspace{0.3cm}
    \textcolor{azul}{\Large \textbf{Instituto Superior Universitario Tecnológico del Azuay \\}}
    \vspace{0.3cm}
    \textcolor{azul}{\Large \textbf{Tecnología Superior en Big Data}}
    
    % 1. ---------------- TEMA -------------------------
    
    {\Large\textbf{\Tema}}
    
    % 2. ---------------- AUTOR(ES) -------------------------
    \textcolor{azul}{\large \textbf{\EtiquetaAutores} \\}
    \vspace{0.3cm}
    {\large \Alumno}

    % 3. ---------------- MATERIA -------------------------
    \textcolor{azul}{\large \textbf{Materia:} \\}
    \vspace{0.3cm}
    {\large \Materia}


    % 3. ---------------- DOCENTE -------------------------
    \textcolor{azul}{\large \textbf{Docente:} \\}
    \vspace{0.3cm}
    {\large \Docente}


    % 3. ---------------- Ciclo -------------------------
    \textcolor{azul}{\large \textbf{Ciclo:} \\}
    \vspace{0.3cm}
    {\large \Ciclo}


    % 3. ---------------- FECHA -------------------------
    \textcolor{azul}{\large \textbf{Fecha:} \\}
    \vspace{0.3cm}
    {\large \Fecha}

    % 3. ---------------- PERIODO -------------------------
    \textcolor{azul}{\large \textbf{Periodo Académico:} \\}
    \vspace{0.3cm}
    {\large \Periodo}
 
\end{titlepage}


    \section*{\centering Taller de ejercicios - Límites}
        \textbf{Resolver los siguientes ejercicios:}

        \begin{enumerate}
            \item Estime el valor del límite haciendo una tabla de valores, compruebe su trabajo con una gráfica:
            \item Complete la tabla de valores (a cinco lugares decimales), y use la tabla para estimar el valor del límite:
            \item Para la función $f$ cuya gráfica nos dan, exprese el valor de la cantidad dada si existe; si no existe, explique por qué:
            \item Use la tabla de valores para estimar el valor del límite. A continuación, use una calculadora gráfica para confirmar gráficamente sus resultados:
            \item Evalúe el límite y justifique cada paso al indicar las leyes de límites apropiadas:
            \item Evalúe el límite si existe:
            \item Encuentre el límite, si existe. Si el límite no existe, explique por qué:
            \item Sea:
            \item Sea:
            \item Resuelva los siguientes límites al infinito:
        \end{enumerate}

\end{document}